\documentclass[UTF8]{ctexart}
\usepackage{geometry}
\geometry{left=2.5cm,right=2.5cm,top=2.5cm,bottom=2.5cm}
\usepackage[raggedright]{titlesec}

\title{基于项目的云制造服务组合优化问题研究}
\date{}
\begin{document}
\maketitle
\section{绪论}
	\subsection{研究背景和意义}
		\subsubsection{云制造的研究背景和意义}
		\subsubsection{项目调度的研究背景和意义}
	\subsection{研究现状}
		\subsubsection{云制造的研究现状}
		\subsubsection{项目调度的研究现状}
	\subsection{研究内容和技术路线}
		\subsubsection{研究内容}
		\subsubsection{技术路线}
	\subsection{论文内容及章节安排}

\section{基于项目的云制造服务组合优化问题相关技术分析}
	\subsection{项目调度发展现状}
	\subsection{云制造环境中的资源共享}
	\subsection{服务发现与匹配技术}
	\subsection{服务组合优化技术}
	\subsection{案例分析}
	\subsection{本章小结}

\newpage

\section{云制造环境下的服务组合优化问题}
	\subsection{引言}
	\subsection{问题建模}
		\subsubsection{问题描述}
		\subsubsection{模型建立}
		\subsubsection{问题模型分析}
	\subsection{求解算法}
		\subsubsection{求解算法分析}
		\subsubsection{规划方法}
		\subsubsection{进化算法}
	\subsection{数据生成}
	\subsection{结果分析}
	\subsection{本章小结}

\section{云制造服务组合优化的若干扩展问题}
	\subsection{引言}
	\subsection{总服务数量随时间变动的问题模型}
	\subsection{多项目服务组合优化问题}
	\subsection{多目标的云制造服务组合优化问题}
	\subsection{本章小结}

\newpage

\section{云制造服务组合优化的动态调度}
	\subsection{引言}
	\subsection{动态调度研究现状}
		\subsubsection{动态调度的分类}
		\subsubsection{动态调度的策略}
	\subsection{服务组合优化的动态调度}
	\subsection{实例分析}
	\subsection{本章小结}

\section{总结和展望}
	\subsection{总结}
	\subsection{展望}


\end{document}